\documentclass{article}
\usepackage{listings}
\usepackage{amsthm}
\usepackage{amsfonts}
\usepackage{hyperref}
\usepackage{amssymb}
\usepackage{amsmath,array}


\title{User Guide}
\author{Andrew Ye and James Ross}
\date{April 2024}

\begin{document}
\maketitle
\newpage
\tableofcontents
\newpage

\section{Introduction}
This project was to program a way to row reduce the matrix, find the inverse of a matrix, multiply matrices, and add matrices. 

\section{Accessing the Project}
Navigate to \href{https://github.com/AndrewYe12/RREF/blob/main/all_together.py}{RREF} repository. Download the file \(all\_together.py\) and \(execute.py\). 
Create an instance of the matrix class by assigning a matrix to \(matrix(\text{your matrix goes here})\) inside the \(execute.py\) file. Now you can apply methods to this instance.


\section{Class Methods}
We created a matrix class in order to implement our goal.
\begin{center}
\begin{tabular}{||c|c||}
    \hline
    Operation & Function Name \\ [0.5ex]
    \hline\hline
    Row Reduced Echelon Form & .rref() \\
    Inverse Matrix & .inverse() \\
    Represent each entry as a fraction & .fraction() \\
    Multiply Matrics & \(*\) \\
    Add Matrics & \(+\) \\ [1ex]
    \hline 
\end{tabular}
\end{center}
Note that each method returns an instance of the matrix class, so to access the matrix itself you will have to add \(.matrix\) to the end of the instance. 
Also, the \(.fraction\) method returns a list of list of strings, so \(.fraction()\) should be the last method applied to whatever operation you are applying to the
matrix. 

\section{.rref()}
This row reduces the matrix to echelon form. This method is a bijection between \(\mathbb{R}^{m \times n}\) and \(\mathbb{R}^{m \times n}\)

\section{.inverse()}
This method returns the matrix \(B\) such that \(rref(A) = BA\) where \(A\) is the input matrix.
Thus when \(A\) is square and invertible, it will return the inverse of \(A\) where \(I = rref(A) = BA\). This method is a bijection between \(\mathbb{R}^{m \times n}\) and \(\mathbb{R}^{m \times m}\)

\section{.fraction()}
Returns the matrix as another matrix except each entry is a string in fraction form. 

\section{+}
Matrix addition between two instances of matrix class. \(f:\mathbb{R}^{m\times n} \times \mathbb{R}^{m \times n} \rightarrow \mathbb{R}^{m \times n}\).

\section{*}
Matrix multiplication between two instances of matrix class. \(f:\mathbb{R}^{m\times n} \times \mathbb{R}^{n \times p} \rightarrow \mathbb{R}^{m \times p}\).


\end{document}